\documentclass[11pt,onecolumn]{article}
\author{ $Srinivasan Kannan (alias) Ka.Shrinivaasan (alias) Shrinivas Kannan$ \\
$Ph: 9789346927, 9003082186, 9791165980$ \\
$Krishna iResearch OpenSource Products: http://sourceforge.net/users/ka\_shrinivaasan,$ \\
$ZODIAC DATASOFT: https://github.com/shrinivaasanka/ZodiacDatasoft$ \\
$https://www.ohloh.net/accounts/ka\_shrinivaasan$ \\
$Research Website: https://sites.google.com/site/kuja27/$ \\
$(ka.shrinivaasan@gmail.com, shrinivas.kannan@gmail.com, kashrinivaasan@live.com)$ 
}
\title{Arrow's Theorem, Circuit For Democracy and Psuedorandom Choice and P Versus NP - Draft}
\usepackage{ucs}
\usepackage{amsmath}
\usepackage[margin=1in]{geometry}
\usepackage{amsthm}
\usepackage{amsfonts}
\usepackage{amssymb}
\usepackage[pdftex]{color,graphicx}
\usepackage{abstract}
\newtheorem{defn}{Definition}
\newtheorem{theorem}{Theorem}
\begin{document}
\thispagestyle{empty}
\pagestyle{empty}
\maketitle
\begin{onecolabstract}
In this draft article a circuit for majority voting that simulates the real-world electors is constructed and its relation to Arrow's theorem and implications for P Vs NP is described. This is a typeset version that unifies already existing handwritten notes and drafts.
\end{onecolabstract}

\section{Pseudorandom Choice and Circuit for Democracy}
\begin{enumerate}
\item Conventional Majority circuit that outputs 1 if atleast half the inputs are one is chosen

\item Inputs to this majority circuit are SAT circuits for each voter in a democracy. 

\item Wishlist of each of the voter is reduced to a CNF formula and a SAT circuit is constructed for it. 

\item The variables and clauses in the voter SAT are expectations of each of the voter that is assigned 1 or 0 by the voter while deciding to vote for or against.

\item Thus above Majority circuit augmented with SAT oracles or circuits is in NP. This reduction from 3-SAT is rather trivial.

\item Above circuit makes democracy an NP-complete problem.

\item Using the arguments for $100\%$ convergence case of P(good) series (mentioned in bibliography links below) there exists a Pseudorandom generator based Choice in P or NC that achieves the same objective thereby creating an intriguing anomaly.

\item In the P(good) series if both LHS(Pseudorandom choice) and RHS(Majority voting) are $1$ (zero error), then there is a polytime algorithm (pseudorandom choice) to the NP problem of Majority voting which is a counterexample. One way functions exist when the series does not converge and do not exist when series converges to $1$.Thus $P \neq NP$ if and only if there is no perfection in voting (and in general any entity or process in universe). 

\item Moreover, finding even a single perfect voter who
never makes an error itself looks to be intractable. Infact this circuit requires all voter SATs to be perfect for the series to converge to $1$ thereby placing a tighter restriction than mere intractability. 
\end{enumerate}

\section{Boolean function sensitivity, Condorcet Elections, Arrow's Theorem and P(Good) series}
\begin{enumerate}
\item $https://sites.google.com/site/kuja27/ImplicationRandomGraphConvexHullsAndPerfectVoterProblem_
2014-01-11.pdf$ on decidability of existence of perfect voter and the probability series for a good choice of $https://sites.google.com/site/kuja27/CircuitForComputingErrorProbabilityOfMajorityVoting_2014.pdf$ are related to already well studied problems in social choice theory but problem definition is completely different. 

\item Arrow's theorem of social choice for an irrational outcome in condorcet election of more than 2 candidates and its complexity theory fourier analysis proof [GilKalai] are described in $www.cs.cmu.edu/~odonnell/papers/analysis-survey.pdf$. 

\item Irrational outcome is a paradox where the society is "confused" or "ranks circularly" in choice of a candidate in multipartisan condorcet voting. Rational outcome converges to 91.2\% with a possibility of 8.8\% irrational outcome. 

\item What is perplexing is the fact that this seems to contravene guarantee of unique restricted partitions described based on money changing problem and lattices in $https://sites.google.com/site/kuja27/Sc
hurTheoremMCPAndDistinctPartitions_2014-04-17.pdf$ and $https://sites.google.com/site/kuja27/
IntegerPartitionAndHashFunctions_2014.pdf$ which are also for elections in multipartisan setting (if condorcet election is done). 

\item Probably a "rational outcome" is different from "good choice" wh
ere rationality implies without any paradoxes in ranking alone without giving too much weightage to the "goodness" of a choice by the elector. Actual real-life elections are not condorcet elections where NAE tuples are generated. 

\item It is not a conflict between Arrow's theorem as finding atleast 1 denumerant in multipartisan voting partition is NP-complete (as it looks) - which can be proved by ILP as in point 20 of $https://sites.google.com/site/kuja27/IntegerPartitionAndHashFunctions_2014.pdf$ - the assumption is that candidates are not ranked; they are voted independently in a secret ballot by electors and they can be more than 3. The elector just chooses a candidate and votes without giving any ordered ranking for the candidates which makes it non-condorcet. 

\item Moreover Arrow's theorem for 3 candidate condorcet election implies a non-zero probability of error in voting which by itself prohibits a perfect voting system. 

\item If generalized to any election and any number of candidates it could be an evidence in favour of P != NP by proving that perfect voter does not exist and using democracy Maj+SAT circuits and P(Good) probability series convergence (As described in handwritten notes and drafts:
\begin{enumerate}
\item $http://sourceforge.net/projects/acadpdrafts/files/ImplicationGraphsPGoodEquationA
ndPNotEqualToNPQuestion_excerpts.pdf/download$, 
\item $https://sites.google.com/site/kuja27/ImplicationRandomGraph
ConvexHullsAndPerfectVoterProblem_2014-01-11.pdf$,
\item $https://sites.google.com/site/kuja27/LowerBoundsForMajorityVotingPseudorandomChoice.pdf$,
\item $https://sites.google.com/site/kuja27/CircuitForComputingErrorProbabilityOfMajorityVoting_2014.pdf$
\item $https://sites.google.com/site/kuja27/PhilosophicalAnalysisOfDemocracyCircuitAndPRGChoice_2014-03-26.pdf$). 
\end{enumerate}

\item But it is not known that if Arrow's theorem can be generalized for infinite number of candidates as above and whether such an electoral system is decidable. 

\item The possibility of a circular ranking in 3-condorcet election implies that there are some scenarios where voter can err though not exactly an error in making a "good choice" (or Perfect Voter Problem is decidable in case of 3 candidates condorcet election).

\item Error by a Voter SAT circuit implies that voter votes 0 instead of 1 and 1 instead of 0. This is nothing but the sensitivity of the voter boolean function i.e number of erroneous variable assignments 
by the voter that change the per-voter decision input to the Majority circuit. 

\item Thus more the sensitivity or number of bits to be flipped to change the voter decision, less the probability of error by the voter. If sensitivity is denoted by s, 1/q is probability that a single bit is flipped and probability of error by the voter is p then $p = k/q^{s}$ for some constant k which is derived by the conditional probability that Pr[m bits are flipped] = Pr[m-th bit is flipped/(m-1) bits already flipped]*Pr[(m-1) bits are flipped] = $1/q$
*$1/q^{m-1}$ = $1/q^{m}$ (and m=s) . 

\item This expression for p  can be substituted in the Probability series defined in $https://sites.google.com/site/kuja27/CircuitForComputingErrorProbabilityOfMajorityVoting_2014.pdf$. Probability of single bit is 1/q if the number of variables across all clauses is q and each variable is flipped independent of the other.


\end{enumerate}

\section {Majority Circuit with SAT input voters - Illustration with a P(good) series example}
\includegraphics[scale=0.5]{MajSATcircuit_Screenshot_2014-09-17-143859.png} 

\section{Acknolwdgement}

I dedicate this article to God.

\section{Bibliography}
\begin{thebibliography}{99}

\bibitem{pa} The series convergence is illustrated in
$http://sourceforge.net/p/asfer/code/HEAD/tree/cpp-src/miscellaneous/MajorityVotingErrorProbabilityConvergence.JPG$

\bibitem{pa} Algorithms for Intrinsic Merit of a Document - Recursive Gloss Overlap Algorithm for wordnet subgraph - $http://arxiv.org/abs/1006.4458$

\bibitem{pa} Slides iluustrating WordNet subgraph or Definition Graph Construction using Recursive Gloss Overlap Algorithm - $https://sites.google.com/site/kuja27/PresentationTAC2010.pdf?attredirects=0$

\bibitem{pa} Document Summarization from WordNet subgraph obtained by Recursive Gloss Overlap - $https://sites.google.com/site/kuja27/DocumentSummarization_using_SpectralGraphTheory_RGOG
raph_2014.pdf$

\bibitem{pa} Primitive implementation of the above at: $http://sourceforge.net/p/asfer/code/HEAD/tree/python-src/InterviewAlgorithm/$

\bibitem{pa} TAC 2010 - Update Summarization using Interview Algorithm
- $http://www.nist.gov/tac/publications/2010/participant.papers/CMI_IIT.proceedings.pdf$

\bibitem{pa} WordNet - $http://wordnet.princeton.edu/$

\bibitem{pa} Equating Majority Voting and Pseudorandom Choice - $https://sites.google.com/site/kuja27/LowerBoundsForMajorityVotingPseudorandomChoice.pdf?attredirects=0$

\bibitem{pa} Probability of good majority choice Series Derivation - 
$https://sites.google.com/site/kuja27/CircuitForComputingErrorProbabilityOfMajorityVoting_2014.pdf$

\bibitem{pa} Axiom of Choice and Majority Voting
$https://sites.google.com/site/kuja27/IndepthAnalysisOfVariantOfMajorityVotingwithZFAOC_2014.pdf?attredirects=0$

\bibitem{pa} A PRG for Pseudorandom Choice using Chaos Theory Attractors
$https://sites.google.com/site/kuja27/ChaoticPRG.pdf?attredirects=0$

\bibitem{pa} Interview Algorithm and Interactive Proof
$https://sites.google.com/site/kuja27/InterviewAlgorithmInPSPACE.pdf?attredirects=0$

\bibitem{pa} Equivalence of Integer Partitions and Hash Functions (This also gives an alternative proof of NP-completeness of multipartisan democracy using reduction from restricted partition problem Money Changing Problem and Schur theorem for partitions)- $https://sites.google.com/site/kuja27/IntegerPartitionAndHashFunctions_2014.pdf$ 

\bibitem{pa} Handwritten notes: Philosophical analysis of Democracy circuit and Pseudorandom choice
$https://sites.google.com/site/kuja27/PhilosophicalAnalysisOfDemocracyCircuitAndPRGChoice_20140326.pdf?attredirects=0$

\bibitem{pa} Handwritten notes: on Schur Theorem, Restricted Partitions and Hash Functions, NP-Completeness of MultiPartisan Majority Voting
$https://sites.google.com/site/kuja27/SchurTheoremMCPAndDistinctPartitions_20140417.pdf?attredirects=0$

\bibitem{pa} Handwritten notes: An experimental theory of Convex Hull of the logical implication graph and Perfect Voter problem -
$https://sites.google.com/site/kuja27/ImplicationRandomGraphConvexHullsAndPerfectVoterProblem_20140111.pdf$

\bibitem{pa} Handwritten notes: Experimental Notes on Logical Implication Graphs: An experimental logical implication graph model for P and NP - $http://sourceforge.net/projects/acadpdrafts/files/ImplicationGraphsPGoodEquationAndPNotEqualToNPQuestion_excerpts.pdf/download$

\bibitem{pa} A gadget for randomized space filling to simulate some natural phenomena and LP for it -
$https://sites.google.com/site/kuja27/Analysis\%20of\%20a\%20Randomized\%20Space\%20Filling\%20Algorithm\%20and\%20its\%20Linear\%20Program\%20Formulation.pdf$

\bibitem{pa} Social Choice Theory, Arrow's Theorem and Boolean functions - $http://www.ma.huji.ac.il/~kalai/CHAOS.pdf$

\end{thebibliography}
\end{document}

