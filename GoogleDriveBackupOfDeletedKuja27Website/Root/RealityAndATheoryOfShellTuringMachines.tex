\documentclass[11pt,onecolumn]{article}
\author{ Ka.Shrinivaasan (ka.shrinivaasan@gmail.com) }
\title{Levels of Reality and a Theory of Shell Turing Machines}
\usepackage[T1]{fontenc}
\usepackage{ucs}
\usepackage{amsmath}
\usepackage[margin=1in]{geometry}
\usepackage{amsthm}
\usepackage{amsfonts}
\usepackage{amssymb}
\usepackage[pdftex]{color,graphicx}
\usepackage{abstract}
\newtheorem{defn}{Definition}
\newtheorem{theorem}{Theorem}
\begin{document}
\thispagestyle{empty}
\pagestyle{empty}
\maketitle
\begin{onecolabstract}
In this short article a fundamental philosophical question about nature of reality is raised and a new theory of Shell Turing Machines with dimensions is proposed to explain this phenomenon of reality and its multiple levels
\end{onecolabstract}

\section{Introduction}
The physical reality is limited by spacetime dimensions. For example an object on 2-dimension has less knowledge and information than an object on 3-dimension with volume being a distinguishing factor. In this article an effort is made to generalize the concept of dimensions (as not just restricted to spacetime) and integrate that into Turing machines.

\section{Few Definitions Associated With Shell Turing Machine}

\begin{defn}
We define a logical dimension as any new variable that adds knowledge to an existing algorithm (or) algorithm gets more computational power and emits more knowledge when this dimension is added to existing set of dimensions. 
\end{defn}

\begin{defn}
Let us define a Shell $TM(n)$ as operating in a logical dimension n where a turing machine TM is value-added with a dimension n as an additional parameter and a limited knowledge domain determined by the dimension(as described by previous definition) 
\end{defn}

\begin{defn}
We define Reality as a set of entities and their interactions and events happening within a maximum box universe of logical dimension N (which is not engulfed by another box universe and there is no $M > N$)
\end{defn}

\begin{defn}
We define Unreality as a set of entities and their interactions and events happening within a box universe of logical dimension P (which can be engulfed by another box universe and there is some $M > P$). Important thing to note is that we realize that something is unreal only if we exit the Unreality and events in Unreality can only be observed and not written-to by a universe in which unreality is embedded.(Example: waking up from a dream). Thus we can rephrase Unreality as a Reality in a lower logical dimension.
\end{defn}

\begin{defn}
Let us define Unreal reality as a set of entities and events involving those entities which is real in some universe(or)context of lower dimension but unreal in some other universe(or)context of higher dimension 
\end{defn}

\begin{defn}
Let us define Real unreality as a set of entities and events involving those entities which is unreal in some universe(or)context of higher dimension but real in some other universe(or)context of lower dimension
\end{defn}

\section{Some examples}
\begin{enumerate}
\item A colloquial reallife example: If we consider a movie as a shell, the sequence of events in a movie can be encoded by  turing machine states and same physical laws apply. But the audience of the movie which can be encoded as another turing machine cannot interact with the movie turing machine(movie turing machine is read-only). Thus we have a pseudoreality embedded within a reality. Extrapolating this, a movie within a movie is a pseudopseudoreality and so on. We can say that a movie turing machine is enacted in a universe of logical dimension of atleast 1 less than the reality turing machine. 

\item Another example: When two mirrors are facing each other, we see an avalanche of mirror embedded  within mirror embedded within mirror and so on, on each physical mirror.
\end{enumerate}



\section{Paradox or is it?}

We now ask a question which is an etymological paradox (might sound as an erroneous sentence on first read) and try to answer:
\begin{equation}
Real \quad Unreality \quad and \quad Unreal \quad Reality: \quad are \quad these \quad two \quad equal?
\end{equation}

\section{Reality and Unreality - viewed through a thought experiment}
\begin{enumerate}
\item Let us consider a thought experiment to demonstrate above paradox:
The real life events of a family are videographed and streamed in real-time on to a TV set. Thus TV plays the events in realtime as they happen in reallife (The mirror experiment above can be related to this).Thus same events(or)algorithm are enacted in both reallife and by the TV. TV has a logical dimension atleast 1 less than the physical reality. Physical reality is represented by a Shell turing machine $TM_{1}(N)$ and the movie turing machine(which can be thought of as TV in this example) is represented by a Shell turing machine $TM_{2}(P)$ where $N=P+1$

\item An Example of Unreal reality: above realtime movie is real for the characters within the movie turing machine universe but unreal in audience turing machine. Thus it is real yet unreal though both enact same events (simultaneously almost). Here direction of perception of events is from a higher dimension to a lower dimension.

\item An Example of Real unreality: above realtime movie is unreal for audience turing machine but real for the characters inside the movie turing machine. Thus it is unreal yet real though both enact same events(simultaneously almost).Here direction of perception of events is from lower dimension to higher dimension.

\item Thus we get a tree of shell turing machines where shell of dimension n is a parent of shells of dimension n-1


\item Now we can frame the paradox with logical dimension information as,
\begin{quote}
Is Real(S) Unreality(P) equal to Unreal(Q) Reality(R)?
\end{quote}
where S,P,Q and R are logical dimensions. Thus from above description of some examples real unreality and unreal reality and assuming execution of the same algorithm across dimensions we formalize this as giving a turing machine encoding  as input to the same turing machine and stratify the reality by dimensional information. Important to note here is that we have separated a same Turing Machine executing same algorithm into two Turing Machines $TM_{1}(X)$ and $TM_{2}(Y)$ but differing only in their dimensions X and Y ($X \geq Y$).
\end{enumerate}

\section{What is Equality of Real Unreality and Unreal Reality?}

Question of equality loses its meaning unless dimensions and algorithm executed by the two Turing Machines are spelt out clearly and depends on definition of equality
\begin{enumerate}
\item If we define equality as whether two turing machines execute same algorithm then they are equal .
\item If we define equality as whether two turing machines have same logical dimensions (despite executing same algorithm) then they are unequal. Thus the question depends on definition of equality. 
\end{enumerate}

\section{Conclusion}

Real unreality and unreal reality and their equality depend on dimensional information and on definition of equality. Thus the paradox above gives an insight into stratification of Reality

\section{Acknolwdgement}

I dedicate this article to God.

\section{Bibliography}
\begin{thebibliography}{99}
\bibitem{pa} Inspired by a Hindu religious philosophy of Maya where every participatory being in Maya thinks that Maya is real but the Supreme Soul outside the Maya enacts the Maya and thus has more dimensions than the Maya itself not realizable by beings part of Maya and liberation from Maya is called Moksha
\bibitem{pa} Inspired by Turing's Halting problem undecidability result -particularly the gadget used for it which inputs an encoding of a Turing Machine to the same Turing Machine
\bibitem{pa} Einstein's quote: A problem cannot be solved by same level of thinking that created it
\end{thebibliography}
\end{document}

