\documentclass[11pt,onecolumn]{article}
\author{ Ka.Shrinivaasan (ka.shrinivaasan@gmail.com) }
\title{Integer partitions and their mapping to hash functions}
\usepackage[T1]{fontenc}
\usepackage{ucs}
\usepackage{amsmath}
\usepackage[margin=1in]{geometry}
\usepackage{amsthm}
\usepackage{amsfonts}
\usepackage{amssymb}
\usepackage[pdftex]{color,graphicx}
\usepackage{abstract}
\newtheorem{defn}{Definition}
\newtheorem{theorem}{Theorem}
\begin{document}
\thispagestyle{empty}
\pagestyle{empty}
\maketitle
\begin{onecolabstract}
This article is a short observation of an interesting relation between Integer Partitions and Hash  Functions and derives a number of possible hash functions based on this relation.
\end{onecolabstract}

\section {Introduction}

A widely used notion of hash function maps a key to value as h(x) = y. If x1  and x2 are two key values and if h(x1) and h(x2) are equal then x1 and x2 are placed in same bucket. Thus a hash table partitions the set of keys to be hashed into sets of buckets of keys having same hash value.  


\section {Generating functions to represent Integer Partitions and Euler's theorem}

First we study how integer partitions are represented and later their mapping to buckets of hash functions . One way is through generating function and applying Euler's theorem. Consider the product: 
\begin{equation}
(1 + x + x^2 + x^3 ...... )(1 + x^2 + x^4 + x^6 .......)(1 + x^3 + x^6 .......)(1 + x^4 + x^8 .......)...
\end{equation}
To illustrate, consider the coefficient of $x^3$. By choosing x from the first parenthesis, $x^2$ from the second, and 1 from the remaining parentheses, we obtain a contribution of 1 to the coefficient of $x^3$. Let the monomial chosen from the i-th parenthesis $1+x^{i}+x^{2i}+x^{3i} · · ·$ in (1) represent the number of times the part $i$ appears in the partition. In particular, if we choose the monomial $x^{i*c_{i}}$ from the i-th parenthesis, then the value i will appear $c_{i}$ times in the partition. Each selection of monomials makes one contribution to the coefficient of $x^n$ and in general, each contribution must be of the form $x^{1*c_{1}} · x^{2*c_{2}} · x^{3*c_{3}} ... = x^{1*c_{1}+2*c_{2}+3*c_{3}.....}$. Thus the coefficient of $x^n$ is the number of ways of writing $n = c_{1} + 2*c_{2} + 3*c_{3} + ......$ where each $c_{i} \geq 0$. Notice that this is just another way to represent an integer partition. As an example $5=1+2+2$ can be written as $5 = 1*1 + 2*2$. Above generating function is an infinite product of geometric series p(n) which is the Euler's partition theorem. 

\section {Number of possible hash functions}
Having arrived at a way to express the integer partitions and parts in a partition, we analyse how integer partitions and hash functions are related. Each hash table partitions the hashed elements into sets of buckets. We can map each of these buckets to a part in an integer partition. Thus if there are x parts in a partition of n elements then there will be x non-empty buckets in the hash table where size of each bucket is equal to the value of the corresponding part in the partition and is thus a one-to-one and onto mapping. If there are m possible hash values then each of these x parts or buckets can be arranged in $mC_{x}$ ways for each partition of n elements. If we aggregate it over all the partitions we get all possible ways of placing an element in a bucket which is nothing but all possible hash functions.

\begin{enumerate}
\item Let m be the number of possible values of hash function h(x)
\item Let n be the total number of elements which will be hashed and placed in buckets
\item Each hash entry would have a linked list of elements hashed on to a hash value for that entry
\item Let lamda(i) be the number of parts in partition i
\item Let p(n) be the partition function
[$lamda(i) \leq m$ and $m \geq n$]

\item Then number of possible hash functions = 
\begin{equation}
\sum_{i=1}^{p(n)} mC_{lamda(i)}
\end{equation}

where lamda(i) which is the number of parts in partition i can be obtained from above generating function for integer partitions as sum of all $c_{i}$'s,

\begin{equation}
\sum_{i=1}^{q} c_{i}
\end{equation}
where the value i will appear $c_{i}$ times in the partition and q is total number of distinct integer in the partition
\end{enumerate}

\section{Acknowledgement}
I dedicate this article to God.

\section{Bibliography}
\begin{thebibliography}{99}
\bibitem{pa} Lectures on Integer Partitions by Herbert Wilf, University of Pennsylvania
\bibitem{pa} This idea was first mentioned by the author in some informal internal email communications at Sun Microsystems when the author worked at Sun Microsystems from 2000-2005(mailid: kannan.srinivasan@sun.com)
\bibitem{pa} Various lecture notes on Generating Functions(Combinatorics)
\end{thebibliography}
\end{document}
